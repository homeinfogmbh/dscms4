\documentclass[a4paper]{article}

% Paketimporte
\usepackage[ngerman]{babel} % Neue Deutsche Rechtschreibung
\usepackage[utf8]{inputenc} % UTF-8 für plattformunabhängige Codierung mit Umlauten
\DeclareUnicodeCharacter{00A0}{ } % Für no-break-spaces
\usepackage{eurosym} % €-Symbol
\usepackage{titlesec} % Textüberschriften anpassen
\usepackage[pdftex]{graphicx} % Für Grafiken
\usepackage{float} 
\usepackage{url} % Für korrekte URL-Formatierung
\usepackage{wrapfig} % Für Abbildungen mit textumlauf
\usepackage{csquotes} % Für Zitate
\usepackage{glossaries} % Für Glossar
\usepackage[markup=nocolor,deletedmarkup=xout]{changes} % Für Änderungen
\usepackage[section]{placeins} % Für floar barrier

% \titleformat{⟨Überschriftenklasse⟩}[Absatzformatierung⟩]{⟨Textformatierung⟩} {⟨Nummerierung⟩}{⟨Abstand zwischen Nummerierung und Überschriftentext⟩}{⟨Code vor der Überschrift⟩}[⟨Code nach der Überschrift⟩]

\titleformat{\chapter}[hang]{\large\bfseries}{\thechapter\quad}{0pt}{}
\titleformat{\section}[hang]{\large\bfseries}{\thesection\quad}{0pt}{}
\titleformat{\subsection}[hang]{\large\bfseries}{\thesubsection\quad}{0pt}{}
\titleformat{\subsubsection}[hang]{\large\bfseries}{\thesubsubsection\quad}{0pt}{}
\titleformat{\paragraph}[hang]{\large\bfseries}{\theparagraph\quad}{0pt}{}

% \titlespacing{⟨Überschriftenklasse⟩}{⟨Linker Einzug⟩}{⟨Platz oberhalb⟩}{⟨Platz unterhalb⟩}[⟨rechter Einzug⟩]

\titlespacing{\chapter}{0pt}{-3em}{6pt}
\titlespacing{\section}{0pt}{6pt}{6pt}
\titlespacing{\subsection}{0pt}{6pt}{6pt}
\titlespacing{\subsubsection}{0pt}{6pt}{6pt}
\titlespacing{\paragraph}{0pt}{6pt}{6pt}

%Toleranzen für Mikrotypographie
\pretolerance=150
\setlength{\emergencystretch}{10em}

% Metadaten
\title{Entwurf zum Projekt \linebreak \enquote{DSCMS4}}
\author{HOMEINFO - Digitale Informationssysteme GmbH}
\date{\today}

% Inhaltsverzeichnis-Konfiguration
\setcounter{tocdepth}{5}
\setcounter{secnumdepth}{5}

% Glossar
\makeglossaries
	
\begin{document}	
	% Titelblatt
	\maketitle
	\pagebreak
	
	% Inhaltsverzeichnis
	\tableofcontents
	\pagebreak
	
	\section*{Vorwort}
	Dieser Entwurf stellt fest, wie das DSCMS4 implementiert werden soll. Er enthält dazu die Beschreibung der notwendigen technischen Komponenten deren Aufbau und Beziehungen zueinander.
	Vermerke und ausstehende Aktionen sind \emph{kursiv} gedruckt.
	\section*{Änderungsliste}
	\begin{itemize}
		\item 05.03.2017
		\begin{itemize}
			\item Erstellung des Dokuments.
		\end{itemize}
	\end{itemize}
	
	\pagebreak
	\section{Komponenten}
	Das DSCMS4 teilt sich als Webanwendung in Frontend und Backend auf.
	\subsection{Frontend}
	Das Frontend des DSMCS4 soll, wie bereits in der Anforderungsspezifikation festgehalten in HTML5 mit CSS3 und JavaScript implementiert werden.
	Eine serverseitige Bearbeitung des Frontends ist ausgeschlossen.
	TODO: Implementationsbeschreibung durch Raphael Haupt.
	\subsection{Backend}
	Das Backend soll gemäß der Anforderungsspezifikation auf einem Linux-Server lauffähig sein und als HIS-Modul implementiert werden.
	Daher ist eine Implementation as UWSGI Dienst in Python3 naheliegend.
	\subsubsection{Datenbanken}
	Um die in der Anforderungsanalyse vorgegebenen Entitäten werden in einer MariaDB Datenbank mit InnoDB Engine wie folgt implementiert.
	TODO: (E)ER Diagramme einfügen.
	\subsubsection{Objektrelationales Mapping}
	TODO: ORM Modelle
	\subsubsection{Controller}
\end{document}
